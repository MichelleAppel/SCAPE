\section{Introduction}
Visual cortical prostheses offer a promising path to restore vision in individuals with severe visual impairment by directly stimulating populations of neurons in visual cortex. These devices, such as the Utah array \cite{Maynard1997, Normann1999,Schmidt1996} and polyimide-based, ultraflexible cortical implants such as those developed by Neuralink and others \cite{Musk2019,Zhao2023,Orlemann2024}, aim to bypass damaged retinal pathways and directly encode visual information into the brain through early cortical sites \cite{Fernandez2021}. By stimulating cortical neurons in a spatially organized manner, these implants can evoke perceptual phosphenes that correspond to visual stimuli according to their retinotopy \cite{Chen2020}.

However, a central challenge in designing effective cortical prostheses is the limited number of stimulating electrodes, which fundamentally constrains the spatial resolution of visual information. This bottleneck necessitates careful consideration of how visual stimuli are processed and represented before delivery to the implant \cite{Lozano2020,vanderGrinten2024,Beyeler2022,vanSteveninck2022}.

Current approaches to visual encoding for cortical implants often rely on uniform spatial filtering techniques, such as Sobel or Canny edge detection, to reduce the dimensionality of visual input. While these methods can help manage the high spatial resolution of natural images, they neglect implant-specific sampling density. This mismatch can produce oversmoothing in regions where electrode coverage is high, reducing available detail, or introduce spurious structure in sparse regions that cannot be faithfully represented by the implant \cite{Kasowski2022,Han2021,Relic2022}.

In this work we introduce SCAPE (Shift variant Cortical prosthesis Adaptive Phosphene Encoding), an adaptive encoding framework that tailors spatial filtering to the local resolvability of each implant configuration. SCAPE first estimates the sampling density from electrode or phosphene positions using analytic magnification models or kernel density estimation. It then maps density to a local spatial scale via Nyquist principles and applies shift variant filtering, implemented here as a difference of Gaussians, to the input image before phosphene rendering.

The main contributions of this paper are:
\begin{itemize}
  \item A principled method for local sampling density estimation and shift variant spatial filtering tailored to cortical implant layouts.
  \item Comprehensive evaluation of SCAPE in simulation across multiple electrode configurations, including high density, Utah array, Neuralink-type shanks, and receptive field-based schemes.
  \item Benchmarking performance with representational similarity analysis, and reconstruction accuracy of an Attention UNet decoder.
\end{itemize}
