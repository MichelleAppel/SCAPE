\section{Discussion and Conclusion}

\subsection{Summary of Contributions}
This work introduced SCAPE, a principled framework for encoding visual information in cortical prostheses that adapts spatial filtering to the local sampling density imposed by electrode layouts. Unlike conventional pipelines that apply uniform filters across the visual field, SCAPE derives a continuous density map from electrode or phosphene locations, converts this into a spatial scale map using Nyquist principles, and applies shift-variant filtering whose kernel width matches the local resolution limit. In an efficient implementation, we demonstrated a separable Difference-of-Gaussians operator that achieves real-time performance while preserving fine detail in dense regions and suppressing clutter in sparse regions.

Across multiple implant schemes and datasets, SCAPE consistently produced more interpretable percepts and preserved the relational structure of stimuli more effectively than nonadaptive baselines. Representational similarity analysis showed that SCAPE maintained pixel-space similarity geometry with higher fidelity, while reconstruction experiments revealed superior decoder performance on both intensity- and perceptual-based metrics. These improvements were observed consistently across natural scenes, faces, and indoor–outdoor environments, underscoring the robustness of the approach.

Together, these findings establish SCAPE as a general and efficient strategy for cortical implant encoding. By explicitly linking electrode density to filter scale, SCAPE provides a flexible foundation that can be integrated with existing simulators, extended with alternative kernel families, and adapted to patient-specific layouts. The framework therefore advances both the methodological toolkit for prosthetic vision research and the practical feasibility of density-aware encoding in future clinical systems.


\subsection{Relation to Prior Work}
The present work builds on several decades of research in prosthetic vision encoding while addressing a gap that has remained largely unfilled. Early heuristic pipelines emphasized contrast enhancement and edge extraction, which offered computational simplicity but treated the visual field uniformly. These methods were unable to balance the competing needs of preserving detail in regions of dense sampling and avoiding clutter where electrodes were sparse. More recent learned encoders have demonstrated that end-to-end optimization can improve perceptual quality, particularly when integrated with differentiable phosphene simulators. However, these approaches typically apply spatial filtering in a global manner, optimizing parameters across the full image rather than modulating them according to local sampling constraints.

SCAPE complements these earlier directions by introducing a middle ground between heuristic and fully learned solutions. Like heuristic methods, it is transparent, lightweight, and efficient, making it suitable for real-time use and hardware-constrained environments. At the same time, SCAPE is principled and general, deriving its adaptivity from electrode density and Nyquist sampling theory rather than hand-tuned heuristics. This makes it more systematic than early encoders and more interpretable than deep learned encoders, while remaining compatible with both paradigms.

Finally, SCAPE integrates naturally with recent differentiable simulation frameworks such as that of van der Grinten \emph{et al.}, which allow gradient-based optimization of encoders in realistic implant layouts. Embedding SCAPE into such pipelines enables hybrid approaches in which local density-driven filtering forms the initial encoding stage, while downstream networks can further refine or task-optimize the representation. In this sense, SCAPE provides a bridge between the transparency of heuristic designs and the flexibility of end-to-end learning, offering a principled foundation on which both research and clinical translation can build.

\subsection{Clinical and Practical Implications}
A central challenge in cortical prosthetic vision is that electrode arrays sample the visual field inhomogeneously. Uniform filtering strategies ignore this variation, often leading to oversmoothing in regions of high density and distracting clutter where coverage is sparse. By explicitly linking electrode density to filter scale, SCAPE addresses this mismatch and ensures that each region of the visual field is represented with the appropriate level of detail. This adaptivity has direct practical importance: it improves the interpretability of phosphene patterns, reduces spurious activations, and preserves fine structure where the implant supports it.

The benefits extend beyond simulated percepts. More faithful preservation of local structure implies that functional tasks such as object recognition, face perception, and scene navigation could become easier for users, since relevant features would remain distinguishable rather than being obscured by noise or excessive simplification. The strong performance of SCAPE reconstructions, especially on structured domains such as faces, suggests that the framework may support higher-level tasks critical for social interaction and daily functioning.

Equally important is the computational efficiency of SCAPE. The separable Difference-of-Gaussians implementation provides real-time performance on modest hardware, making the method compatible with the stringent latency and power constraints of implant systems. This efficiency ensures that the advantages of density-adaptive encoding are not merely theoretical but can be realized in practice on embedded devices and potentially integrated into future clinical systems.

Together, these properties position SCAPE as a practical foundation for next-generation cortical implant pipelines. By improving the clarity and usability of prosthetic percepts while remaining lightweight and transparent, SCAPE contributes to bridging the gap between laboratory simulations and clinical translation.

\subsection{Limitations}
While SCAPE provides a principled and efficient framework for adaptive phosphene encoding, several limitations should be acknowledged. First, all results were obtained in simulation. Although the simulator models important aspects of cortical stimulation and phosphene appearance, it cannot fully capture the variability and subjective experience of human percepts. Clinical evaluation will therefore be essential to validate whether the improvements observed in simulation translate into functional benefits for implant users.

Second, we evaluated usability indirectly through decoder reconstructions. Decoder performance offers a convenient proxy for how much visual information is preserved, but it is not equivalent to behavioral outcomes such as recognition accuracy, navigation performance, or reading speed. Direct psychophysical and clinical studies will be required to confirm that SCAPE leads to measurable improvements in real-world tasks.

Third, our analysis focused entirely on static images. We did not account for the temporal dynamics of stimulation, such as persistence, adaptation, or temporal integration of phosphenes. These dynamics strongly influence perceptual quality and functional performance in real use, and incorporating them into adaptive encoding remains an important direction for future work.

Finally, SCAPE is presented here as a heuristic grounded in sampling theory and density adaptation rather than as a fully optimized end-to-end solution. This makes it interpretable and efficient but also leaves open questions about how it compares to task-specific or learned encoders when evaluated in real use cases. Future behavioral experiments will be key to assessing the practical utility of SCAPE and to refining its role in clinical prosthetic vision pipelines.

\subsection{Future Directions}
Several avenues of research follow naturally from this work. A priority is to validate SCAPE in behavioral and eventually clinical settings. While simulation provides a controlled environment for development, only user studies can reveal whether the improved preservation of local structure translates into measurable benefits for recognition, navigation, and daily activities. Integrating SCAPE into immersive VR-based prosthetic vision platforms would provide a tractable next step toward such validation.

A second direction is to incorporate temporal dynamics of phosphene perception. The present study considered only static images, but in practice users encounter dynamic environments where persistence, adaptation, and temporal integration shape perceptual experience. Extending SCAPE to account for temporal structure—by adapting filter scale not only across space but also across time—may further improve perceptual clarity and stability.

Beyond these extensions, SCAPE offers a flexible foundation for hybrid approaches. Its density-driven filtering could be combined with human-in-the-loop optimization to tailor encoding to subjective reports, or integrated into differentiable pipelines where the adaptive filtering stage is refined jointly with a downstream decoder. More sophisticated kernel families, such as orientation-tuned Gabors or steerable filters, could also be explored to capture feature selectivity beyond isotropic contrast.

Finally, the computational efficiency of SCAPE makes it a promising candidate for deployment on resource-limited implant hardware. Future work should evaluate performance on embedded processors or FPGA-based platforms, ensuring that the benefits of adaptive encoding are practical under the latency and power constraints of clinical devices.

Together, these directions highlight how SCAPE can evolve from a principled static encoding framework into a broader platform that adapts across space, time, tasks, and hardware, ultimately supporting translation from simulation to patient use.

\subsection{Conclusion}
We introduced SCAPE, a density-adaptive framework for cortical prosthetic vision that links electrode layout to spatial filtering through Nyquist-based scale mapping. By implementing this principle with an efficient separable Difference-of-Gaussians, SCAPE preserves detail where electrode coverage is dense and suppresses clutter where it is sparse. Across multiple datasets and implant schemes, SCAPE consistently improved representational fidelity and reconstruction quality relative to nonadaptive baselines. These findings demonstrate that principled adaptation to local sampling density can substantially enhance the clarity and interpretability of simulated prosthetic percepts. SCAPE is lightweight, general, and compatible with existing simulators, making it well suited as a foundation for future patient-specific encoding strategies. By bridging the gap between heuristic efficiency and adaptive precision, SCAPE provides a promising step toward next-generation cortical implant pipelines and lays the groundwork for behavioral and clinical validation.
