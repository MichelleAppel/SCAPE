\section{Discussion and Conclusion}

\subsection{Summary of Contributions}
This work introduced SCAPE, a principled framework for encoding visual information in cortical prostheses that adapts spatial filtering to the local sampling density imposed by electrode layouts. Unlike conventional pipelines that apply uniform filters across the visual field, SCAPE derives a continuous density map from electrode or phosphene locations, converts this into a spatial scale map using Nyquist principles, and applies shift-variant filtering whose kernel width matches local resolution limits. In an efficient implementation, we demonstrated a separable Difference-of-Gaussians operator that achieves real-time performance while preserving detail in dense regions and suppressing clutter in sparse ones.

Across multiple implant schemes and datasets, SCAPE consistently produced more interpretable percepts and preserved relational structure more effectively than nonadaptive baselines. Both representational similarity analysis and reconstruction experiments confirmed these gains across natural scenes, faces, and indoor–outdoor environments. These findings establish SCAPE as a general and efficient strategy for cortical implant encoding that can be integrated with existing simulators, extended with alternative kernels, and adapted to patient-specific layouts.

\subsection{Relation to Prior Work}
Research on prosthetic vision encoding spans from early heuristics to recent learned approaches. Early pipelines emphasized contrast enhancement and edge extraction, which offered computational efficiency but treated the visual field uniformly, limiting their ability to balance detail in dense regions and clutter in sparse ones. Learned encoders have demonstrated improved perceptual quality, particularly when combined with differentiable simulators, but typically apply spatial filtering globally rather than adapting to local sampling constraints. Simulation studies further highlight that excessive detail can overwhelm sparse layouts, while overly uniform filtering can erase useful structure \cite{Han2021,Kasowski2022}.

SCAPE complements these directions by offering a middle ground: lightweight and interpretable like heuristics, but grounded in electrode density and Nyquist theory rather than hand-tuned parameters. It also integrates naturally with modern differentiable simulators \cite{vanderGrinten2024}, enabling hybrid pipelines where density-aware preprocessing forms the first stage and downstream networks refine the encoding.

\subsection{Clinical and Practical Implications}
Electrode arrays sample the visual field inhomogeneously, yet most encoding pipelines ignore this constraint. By linking electrode density to filter scale, SCAPE ensures that each region is represented at an appropriate level of detail. This adaptivity improves interpretability of phosphene patterns, reduces spurious activations, and preserves fine structure where the implant supports it. Such improvements are directly relevant for functional tasks such as object recognition, face perception, and scene navigation—benchmarks in clinical evaluation of prosthetic vision \cite{Stingl2015,Fernandez2021}. 

SCAPE’s separable DoG implementation also achieves real-time performance on modest hardware, making it suitable for the stringent power and latency constraints of implantable systems. These properties position SCAPE as a practical foundation for next-generation cortical implant pipelines, bridging the gap between simulation and clinical translation.

\subsection{Limitations}
Several limitations must be noted. First, all results were obtained in simulation, which cannot capture the full variability of human percepts. Clinical evaluation will be essential to assess whether the observed improvements translate into functional gains. Second, usability was evaluated indirectly through decoder reconstructions, which provide a convenient proxy but not behavioral outcomes such as recognition or navigation. Third, our study focused solely on static images and spatial encoding. Temporal dynamics such as persistence, adaptation, and fading of phosphenes \cite{Schmidt1996,Bak1990,Dobelle1974,Barlett2008,Fernandez2021,Beauchamp2020} were not modeled, though recent simulators now incorporate these mechanisms \cite{vanderGrinten2024}. Finally, SCAPE is presented as a principled but non-learned approach, leaving open how it compares to task-optimized or human-in-the-loop encoders under behavioral evaluation.

\subsection{Future Directions}
Future work should validate SCAPE in behavioral and clinical settings, ideally through immersive VR-based prosthetic vision platforms. Incorporating temporal dynamics of phosphene perception is another priority, extending SCAPE’s adaptivity from space into time. Hybrid approaches also hold promise, where density-driven filtering is refined through task-driven learning or user feedback, and more sophisticated kernels capture orientation or feature selectivity. Finally, SCAPE’s efficiency makes it a strong candidate for embedded deployment, warranting evaluation on processors and FPGAs under realistic power and latency constraints.

\subsection{Conclusion}
SCAPE introduces a density-adaptive encoding framework for cortical prosthetic vision that links electrode layout to spatial filtering through Nyquist-based scale mapping. Efficiently implemented with a separable Difference-of-Gaussians, it improves representational fidelity and reconstruction quality while remaining lightweight and compatible with clinical hardware. By grounding adaptivity in electrode density, SCAPE provides a principled alternative to uniform filtering and a flexible foundation for integration with learned or patient-specific encoders, laying the groundwork for next-generation cortical implants that adapt across space today and across time, tasks, and users in the future.
