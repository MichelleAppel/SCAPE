\section{Discussion and Conclusion}

\subsection{Summary of Contributions}
This work introduced SCAPE, a principled framework for encoding visual information in cortical prostheses that adapts spatial filtering to the local phosphene map sampling density imposed by electrode layouts. Unlike conventional pipelines that apply uniform filters across the visual field, SCAPE derives a continuous density map from electrode or phosphene locations, converts this into a spatial scale map using Nyquist principles, and applies shift-variant filtering whose kernel width matches local resolution limits. In an efficient implementation, we demonstrated a separable Difference-of-Gaussians operator that achieves real-time performance while preserving the encoded images’ detail in dense regions and suppressing clutter in sparse ones.

Across multiple cortical implant schemes and datasets, SCAPE consistently produced more interpretable percepts and preserved relational structure more effectively than non-adaptive baselines. Both representational similarity analysis and reconstruction experiments confirmed these gains across natural scenes, faces, and indoor–outdoor environments. These findings establish SCAPE as a general and efficient strategy for phosphene vision encoding that can be integrated with existing neural interface pipelines\cite{Lozano2020}, phosphene simulators\cite{vanderGrinten2024, Fine2024, Beyeler2017}, extended with alternative kernels, and adapted to patient-specific layouts.

\subsection{Relation to Prior Work}
Research on prosthetic vision encoding spans from early heuristics to recent learned approaches. Early pipelines emphasized contrast enhancement and edge extraction, which offered computational efficiency but treated the visual field uniformly, limiting their ability to balance detail in dense regions and clutter in sparse ones. Learned encoders have demonstrated improved perceptual quality, particularly when combined with differentiable simulators, but typically apply spatial filtering globally rather than adapting to local sampling constraints. Simulation studies further highlight that excessive detail can overwhelm sparse layouts, while overly uniform filtering can erase useful structure \cite{Han2021,Kasowski2022}.

SCAPE complements these directions by offering a middle ground: lightweight and more interpretable heuristics, but grounded in electrode density and Nyquist theory rather than hand-tuned parameters. It also integrates naturally with modern differentiable simulators \cite{vanderGrinten2024}, enabling hybrid pipelines where phosphene density-aware preprocessing forms the first stage and downstream networks refine the encoding.

\subsection{Clinical and Practical Implications}
Electrode arrays might sample the visual field inhomogeneously, yet most encoding pipelines ignore this constraint. By linking electrode density to filter scale, SCAPE ensures that each region is represented at an appropriate level of detail. This adaptivity improves interpretability of phosphene patterns, reduces spurious activations, and preserves fine structure where the implant supports it. Such improvements are directly relevant for functional tasks such as object recognition, face perception, and scene navigation—benchmarks in clinical evaluation of prosthetic vision \cite{Stingl2015,Fernandez2021}. Moreover, by providing cleaner and more interpretable inputs from the outset, SCAPE may accelerate training outcomes and give patients a head-start in the process of perceptual learning and adaptation with a new implant \cite{Normann2009, Beyeler2024}.

In order to exploit SCAPE's advantages, accurate phosphene maps will need to be obtained. While high-channel count implants have been demonstrated to generate shape perception \cite{Chen2020}, they come with challenges. Practically, the acquisition of the phosphene maps in human participants relies mostly on manually mapping the generated phosphenes using the user´s report after single or multiple electrode stimulation. This procedure will become greatly time consuming and tedious for the implant user when channel numbers go up to the hundreths of thousands. However, new methodologies based in dimensionality reducion of resting-state neural data promise a greatly accelerated and semi-automated procedure even for high electrode counts \cite{Lozano2024}. These methods, combined with receptive field map priors obtained from anatomical scans \cite{Benson2012, Ribeiro2021} or from aligning functional atlas \cite{Rosenke2021}, together with pre-surgical optimal planning \cite{vanHoof2025}, will help obtaining high-quality phosphene maps estimations that can be directly used (or fine-tuned with new behavioral reports) to directly inform our SCAPE framework.


SCAPE’s separable DoG implementation also achieves real-time performance on modest hardware, making it suitable for the stringent power and latency constraints of implantable systems. These properties position SCAPE as a practical foundation for next-generation cortical implant pipelines, bridging the gap between simulation and clinical translation.

\subsection{Limitations}
Several limitations must be noted. First, all results were obtained in simulation, which cannot capture the full variability of human percepts. Clinical evaluation will be essential to assess whether the observed improvements translate into functional gains. Second, usability was evaluated indirectly through decoder reconstructions, which provide a convenient proxy but not behavioral outcomes such as recognition or navigation. Behavioral studies in VR will be useful to test the perceptual outcomes and to establish whether the improvements observed in simulation may translate into perceptual and functional benefits, and ultimately to assess these effects in real clinical investigation settings.

Third, our study focused solely on static images and spatial encoding. Temporal dynamics such as persistence, adaptation, and fading of phosphenes \cite{Schmidt1996,Bak1990,Dobelle1974,Barlett2008,Fernandez2021,Beauchamp2020} were not used, though recent simulators now incorporate these mechanisms \cite{vanderGrinten2024, Fine2024}. Finally, SCAPE is presented as a principled but non-learned approach, leaving open how it compares to task-optimized or human-in-the-loop encoders\cite{Granley2023} under behavioral evaluation.

\subsection{Future Directions}
Future work should validate SCAPE in behavioral and clinical settings, ideally through immersive VR-based prosthetic vision platforms. Incorporating temporal dynamics of phosphene perception is another priority, extending SCAPE’s adaptivity from space into time. Another direction is to integrate depth information into the encoding pipeline, allowing scale to be modulated not only by electrode density but also by viewing distance and scene geometry. Hybrid approaches also hold promise, where density-driven filtering is refined through task-driven learning or user feedback, and more sophisticated kernels capture orientation or feature selectivity. Finally, SCAPE’s efficiency makes it a strong candidate for embedded deployment, warranting evaluation on processors and FPGAs under realistic power and latency constraints.

\subsection{Conclusion}
SCAPE introduces a density-adaptive encoding framework for cortical prosthetic vision that links electrode layout to spatial filtering through Nyquist-based scale mapping. Efficiently implemented with a separable Difference-of-Gaussians, it improves representational fidelity and reconstruction quality while remaining lightweight and compatible with clinical hardware. By grounding adaptivity in electrode density, SCAPE provides a principled alternative to uniform filtering and a flexible foundation for integration with learned or patient-specific encoders, laying the groundwork for next-generation cortical implants that adapt across space today and across time, tasks, and users in the future.
